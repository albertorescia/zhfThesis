\documentclass[10pt,a4paper]{book}
\usepackage[utf8]{inputenc}
\usepackage[english]{babel}
\usepackage{amsmath}
\usepackage{amsthm}
\usepackage{mathtools}
\usepackage{array}
\usepackage{booktabs}
\usepackage{gensymb}
\usepackage{slashed}
\usepackage{physics}
\usepackage{bbold}
\usepackage{stackengine}
\usepackage{amsfonts}
\usepackage{svg}
\usepackage{amssymb}
\usepackage{graphicx}
\usepackage{geometry}
\usepackage{subcaption}
\usepackage{pdfpages}
\usepackage[numbers,sort&compress]{natbib}

\begin{document}
In the beginning, there was nothing. Then, there was 

\begin{equation}
\begin{split}
\mathcal{L} =& -\frac{1}{4}F_{\mu\nu}F^{\mu\nu}\\ +& i\overline{\psi}D\psi + h.c.\\ +& \overline{\psi}_i y_ij \psi_j \phi + h.c.\\ +& |D_\mu \phi|^2 - V(\phi).
\end{split}
\label{SM Lagrangian}
\end{equation}
Now, before we get too far ahead of ourselves, let us take the time to understand \emph{where} Eq. \ref{SM Lagrangian}, known as the Standard Model Lagrangian, comes from and what it represents. This chapter will be dedicated to this endeavour.
\end{document}