\documentclass[10pt,a4paper]{book}
\usepackage[utf8]{inputenc}
\usepackage{amsmath}
\usepackage{amsfonts}
\usepackage{amssymb}
\usepackage{epigraph}
\usepackage{graphicx}
\author{Alberto Rescia}
\begin{document}

\chapter{The ATLAS Detector}
\epigraph{Shrek: Ogres are like onions. \\
Donkey: They stink? \\
Shrek: Yes. No. \\
Donkey: Oh, you leave 'em out in the sun they get all brown, start sproutin' little white hairs. \\
Shrek: No. Layers. Onions have layers. Ogres have layers. You get it? We both have layers.}


The ATLAS detector, much like a certain emotionally damaged ogre, has many layers. These layers serve the purpose of reconstructing, as much as physically possible, the full kinematics of a given final state. This is no easy task: with several billion bunch crossings per second, state-of-the-art technology\footnote{This was true in the '90s} is needed to meet the challenge. 

In this chapter, we will describe the many layers which compose the ATLAS detector and illustrate how they come together to become the marvel of engineering that we know and love.

\section{The Inner Detector}

\end{document}